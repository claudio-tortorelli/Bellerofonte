\addcontentsline{toc}{chapter}{Bibliografia}
\chaptermark{}
\begin{thebibliography}{breitestes Label}
	\subsection*{Articoli internet}
	 	\bibitem{1.1} Testing Web Services \\ \emph{Neil Davidson} \\ www.webservices.org 
	 	\bibitem{1.2} Web services: un approccio morbido \\ \emph{Luca Balzerani} \\ www.latoserver.it
	 	\bibitem{1.3} Web Services \\ \emph{Elena Bernardi} \\ www.evectors.it
	 	\bibitem{1.4} Web Services \\ \emph{Massimiliano Bigatti} \\ www.mokabyte.it
	 	\bibitem{1.5} Il mistero dei web service\ldots svelato \\ www.idg.it/networking
	 	\bibitem{1.6} What is software testing? And why is it so hard? \\ \emph{James Whittaker} \\ Florida Institute of Technology
	 	\bibitem{1.7} Software debugging, testing and verification \\ \emph{B. Hailpern / P. Santhanam} \\ IBM System Journal, vol. 41, n. 1 2002
	 	\bibitem{1.8} eXtreme Programming Methodology \\  www.extremeprogramming.org
	 	\bibitem{1.9} Quality First - \emph{Mary Hayes} \\ www.informationweek.com
	 		
	\subsection*{Testi}
		\bibitem{2.1} Programming Web Services with SOAP \\ \emph{James Snell} \\ O'REILLY
		\bibitem{2.2} Testing applications on the web \\ \emph{Hung Q. Nguyen} \\ WILEY
		\bibitem{2.3} Computer Networks and Internets with Internet Applications - third edition \\ \emph{Douglas E. Comer}
		\bibitem{2.4} Reti di calcolatori \\ \emph{Larry Peterson / Bruce Davie} \\ Zanichelli
		
	\subsection*{Relazioni, Tesi ed altre pubblicazioni}
		\bibitem{3.1} Il computer che si ripara da solo \\ \emph{Armando Fox / David Patterson} \\ Le Scienze n. 419 Luglio 2003
		\bibitem{3.2} Servizi remoti nei sistemi distribuiti: RPC, thread e panoramica su Corba \\ \emph{Bracciali / Farini / Tortorelli} 
		\bibitem{3.3} Orbilio: una piattaforma Open Source per l'e-learning\\ \emph{Antonio Talarico}\\ Universit� degli Studi di Firenze, Facolt� di SMFN, CDL Scienze dell'Informazione
		\bibitem{3.4} E-learning e Videoconferenza: Teoria ed Applicazione \\ \emph{Dario Di Minno}\\ Universit� degli Studi di Firenze, Facolt� di SMFN, CDL Informatica
		\bibitem{3.5} L'autenticazione e la gestione delle frequenze nella piattaforma Orbilio \\ \emph{Federico Chicchi}\\ Universit� degli Studi di Firenze, Facolt� di SMFN, CDL Informatica
	 	
\end{thebibliography}
	
