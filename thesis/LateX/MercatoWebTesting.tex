\chapter{Breve rassegna del software di web testing}
Elencare a questo punto tutte le possibili alternative tra i tool di web testing \emph{Open Source} e \emph{proprietari} richiederebbe un tempo troppo lungo ed esulerebbe dagli scopi di questa tesi. \'E preferibile evidenziare solo alcuni software fra i pi� rappresentativi, che si distinguono per le soluzioni adottate o per le capacit� di eseguire numerosi tipi di test. Si lascia al lettore il compito di approfondire le qualit� dei singoli tool (peraltro in continua evoluzione), tramite la documentazione presente nei loro siti web. Come punto di partenza si segnala il seguente URL presso il quale � possibile trovare un elenco aggiornato delle principali utility per il web testing: \textbf{http://www.aptest.com/resources.html}.

\begin{itemize}
	\item \textbf{JUnit}: JUnit � ormai uno standard nel software/web testing Java (ma non solo). Esso � costituito da un \emph{framework} che consente di scrivere unit test ripetibili. Da JUnit derivano vari tool di testing che ne sfruttano la struttura, l'idea o le funzionalit�: XMLTestSuite, JWebUnit, HTMLUnit, HTTPUnit ed altri ancora.
	\spazioVer
	\item \textbf{HTTPUnit}: HTTPUnit � una libreria Java disegnata ed implementata da Russell Gold. Consente di accedere ai siti web simulando l'uso di browser e per questo � molto impiegata anche in altri tool. Deriva a sua volta da numerosi pacchetti, tra cui JUnit, SAX, Tidy, XercesImpl e JS. Tra le funzionalit� che mette a disposizione ci sono la compilazione dei form HTML, alcuni aspetti di autenticazione, navigazione negli ipertesti, gestione degli oggetti HTML standard (tabelle, link, immagini, \ldots) e redirezione automatica. Non supporta alla perfezione il \emph{parsing} di alcuni script, funzione che per altro pu� essere disabilitata. Permette invece di leggere le pagine web sia in formato testo che \emph{XML DOM}. Bellerofonte, software presentato in questa tesi, si basa anch'esso su HTTPUnit per implementare i test di base (bench� il verificatore sia libero di utilizzare pacchetti analoghi o pi� specifici per nuovi test). 		
	\spazioVer	
	\item \textbf{Solex}: Solex � un tool per i test funzionali fornito come plug-in della piattaforma di sviluppo Eclipse. Solex fa parte dei tool di testing ``record/replay'"': ha la particolarit� di poter registrare le sessioni client agendo come un server proxy, interponendosi cio� tra il browser ed il server, in ascolto delle richieste e delle risposte HTTP. In seguito le sessioni registrate possono essere rieseguite, con la possibilit� di assegnare dei nuovi parametri alle variabili contenute nei messaggi HTTP.
	\spazioVer
	\item \textbf{PureTest}: PureTest, unita a PureLoad, fornisce una suite di testing piuttosto completa, comprendente un editor grafico di scenari e un \emph{WebCrawler} per l'analisi statica della struttura di un sito web. Oltre che con le funzionalit� disponibili su HTTP, PureTest � compatibile con la maggior parte dei protocolli standard in uso, tra cui NNTP, FTP, SMTP, IMAP, JDBC, LDAP, DNS e JMS. \'E permessa la creazione di scenari sia con il metodo ``record/replay'"' sia con quello ``data driven'"' ovvero editandoli in funzione dei contenuti da testare. PureTest consente l'automazione integrata con Ant di Jakarta ed offre anche un'interfaccia dalla riga di comando per richiamare il tool da altri script.
	\spazioVer
	\item \textbf{Siege}: Siege � un tool Unix per eseguire stress test e ``mettere sotto assedio'"' un web server. Supporta autenticazione, cookies ed i protocolli HTTP e HTTPS.
	\spazioVer
	\item \textbf{TestMaker}: � un tool open source Java rivolto particolarmente al testing delle performance dei Web Service. Dispone di un ambiente grafico di scrittura/esecuzione di test e di un apposito linguaggio di scripting, ispirato a Python, che consente di comunicare agevolmente con SOAP, HTTPS, .NET, JSP. Ha infine la possibilit� di eseguire i test creati in modo concorrente, per simulare un utilizzo condiviso del Web Service.
	\spazioVer
	\item \textbf{JSpider}: � una flessibile implementazione Java di un \emph{web spider},\index{Web spider} ovvero un agente che, muovendosi nel Web (proprio come un ragno sulla ragnatela), raccoglie determinate informazioni sulle pagine incontrate. JSpider consente quindi di verificare automaticamente lo stato di link ed altri elementi web, di controllare se ogni risorsa � raggiungibile, effettuare copie locali di interi siti e, in generale, effettuare test di caricamento e performance.
	\spazioVer
	\item \textbf{Canoo WebTest}: � un tool per i test funzionali basato su HTTPUnit che permette di editare i propri scenari di test direttamente in XML. Esso si integra in modo interessante con ANT, tool di automazione del progetto Jakarta, per creare test suite facilmente combinabili tra loro.
	\spazioVer
	\item \textbf{HTML Validation Service}: � un servizio gratuito di validazione on-line delle pagine HTML e XHTML, offerto dal consorzio W3C per le standardizzazioni nell'ambito Web.
\end{itemize}

